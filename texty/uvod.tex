\subsection{Milý čtenáři!}
Předkládaný text je zamýšlen jako učební pomůcka pro předmět \uv{Kvantová chemie} přednášený studentům čtvrtého ročníku oborů Fyzikální chemie, Anorganická chemie a Molekulární analytická a fyzikální chemie. Předmětem kvantové chemie je aplikace kvantové teorie na problémy řešené v~chemii, zejména pak otázky spojené s~elektronovou strukturou atomů a molekul. U~studentů proto musíme předpokládat alespoň rudimentární znalost kvantové mechaniky v~rozsahu základních přednášek z~Anorganické chemie I~a II, případně Fyziky II či Teoretické chemie. Míra těchto základních znalostí je však u~posluchačů značně odlišná a jeví se nám tak užitečným základní představy a pojmy kvantové mechaniky zopakovat i v~rámci tohoto kurzu, riskujíce tím znuděné pohledy posluchačů pokročilejších. Vlastní oblast kvantové chemie je tak na přednáškách pouze nastíněna a posluchače s~hlubším zájmem ideálně navnadí k~dalšímu studiu. Nejsou tak pokryty formálnější partie kvantové chemie, pokročilejší techniky založené kupříkladu na formalismu druhého kvantování a nepříliš důkladně se zabýváme detaily kvantově-chemických algoritmů. Řada oblastí pouze nastíněných v~našem kurzu je však dále probírána v~navazujících přednáškách z~molekulového modelování, výpočetní chemie či molekulové spektroskopie.

Kvantová chemie může na první pohled působit poněkud odpudivým dojmem. Nevyhneme se v~ní pokročilejším partiím matematiky a může vzbuzovat dojem abstraktnosti a odtrženosti od života. Takový pohled by ale byl mylný, kvantová chemie představuje život sám! Díky ní jsme schopni výhradně ze znalosti základních fyzikálních konstant vypočítat například energie a rozložení náboje v~atomech a molekulách, termodynamické charakteristiky chemických reakcí, rychlost elementárních chemických reakcí, strukturu molekul, kapalin, krystalů či povrchů pevných látek, adsorpční entalpie, molární absorpční koeficienty, NMR posuny...zkrátka vše, co lze v~chemii vyjádřit číslem, dokáže kvantová chemie vypočítat. Tedy, v~principu. Tato vize již stojí za trochu námahy spojené s~pochopením této teorie. 

Nadšení nad krásou chemie ukryté v~jediném vzorci nebylo vždy sdíleno. V~roce 1830 August Comte napsal:

\bigskip

\uv{\textit{Každý pokus o~zavedení matematických metod ke studiu chemických problémů musí být považován za hluboce iracionální a odporující duchu chemie. Jestliže by matematika měla hrát někdy významnou úlohu  v~chemii - úchylka naštěstí málo pravděpodobná - vedlo by to k~rychlé degeneraci této vědy.}}

\bigskip

Ale již o~sto let později (1929) píše zakladatel relativistické kvantové mechaniky Paul Dirac

\bigskip

\uv{\textit{Fyzikální zákony, které jsou nezbytné pro matematickou teorii velké části fyziky a veškerou chemii jsou zcela známy. Jediná potíž tkví v~tom, že přesné použití těchto zákonů vede k~příliš složitým rovnicím, než aby se daly řešit.}}

\bigskip

Přesně to je hlavní mise kvantové chemie. Zákony chemie v~principu známe, ale s~konkrétními aplikacemi se trápíme (a radujeme) již přes 80 let. Věříme, že alespoň část posluchačů se na někdy trnitou cestu kvantové chemie s~námi. Jsme si navíc skoro jisti, že totéž by dnes učinil i~August Comte, pokud by se znovu narodil.
 

V~textu se pravděpodobně nachází větší než obvyklé množství chyb nejrůznějšího druhu, překlepů, gramatických pochybení, typografických nešvarů či docela normálních nesmyslů. Tyto nepořádky padají na hlavu vedoucího autorského kolektivu (PS). Čtenářům budeme za upozornění na chyby vděčni.  



\subsection{Co číst dále?}
Předložený text není ucelenou učebnicí kvantové teorie molekul, nýbrž pouhými sebranými poznámkami k~přednáškám. Jednotlivé kapitoly a obsah docela dobře korespondují s~látkou probíranou na přednáškách\footnote{Přičemž v některých částech látku probíranou na přednáškách pro zájemce poněkud rozšiřujeme.}, jde ale pořád toliko o~fragmenty žádající si rozšiřující informaci. I~přes veškerou snahu může být navíc výklad veden způsobem pro čtenáře málo srozumitelným. Čtenáři proto naléhavě doporučujeme souběžné studium z~dalších pramenů. Níže podáváme stručný komentovaný přehled dostupné literatury.

   	
\subsubsection{Základy kvantové teorie v~chemii}

Studium kvantové chemie, tedy aplikace kvantové teorie na chemické otázky, vyžaduje alespoň rámcovou znalost kvantové teorie jako takové. V~této části proto zmíníme některé z~publikací, které čtenáři pomohou se v~základních principech kvantové teorie zorientovat.

\begin{itemize}

\item David O. Hayward, \textit{Quantum Mechanics for Chemists} RSC, 2002. Tento rozsahem nevelký text lze doporučit jako první text ke studiu kvantové teorie. Je psát čtivým, srozumitelným jazykem a studentovi-začátečníkovi bude výtečným pomocníkem.   
\item Peter Atkins, Julio de Paula, \textit{Fyzikální chemie} VŠCHT Praha, 2013. Kvantová chemie je běžnou součástí základních (tj. bakalářských) kurzů fyzikální chemie. Proto nás nepřekvapí, že asi třetina této nejznámější učebnice fyzikální chemie je věnována právě kvantové chemii. Významná část našeho kurzu je právě v~Atkinsově učebnici pokryta. Učebnice je zároveň pro začínající studenty a typicky je tak dobře srozumitelná.
\item Thomas Engel, \textit{Quantum Chemistry and Spectroscopy} Prentice Hall, 2010. Engelův text je součástí učebnice fyzikální chemie, vyšel ovšem i v~samostatném svazku. Jde opět o~úvodní kurz. Část věnovaná kvantové teorii se autorovi mimořádně povedla, kniha je velmi pěkně zpracována i po grafické stránce. 
\end{itemize}


\subsubsection{Doplňující a rozšiřující literatura}
Mnohé rozšiřující informace nalezne čtenář v~učebnicích Atkinse a Engela, o~kterých byla řeč výše. V~této části okomentujeme pokročilejší texty zaměřené na kvantovou chemii.
 
\begin{itemize}

\item Attila Szabo, Neil S. Ostlund,\textit{Modern Quantum Chemistry} Dover, 1996. Dnes již klasické dílo, které v~sevřené podobě popisuje pokročilé metody kvantové chemie. Kniha se dobře čte a doverská edice je také finančně dosažitelná. Z~dnešního pohledu možná zarazí nepřítomnost teorie funkcionálu hustoty. 
\item Ira N. Levine, \textit{Quantum Chemistry} Pearson/Prentice Hall, 2009. Poctivá učebnice kvantové chemie s~řadou příkladů. Rozsahem velmi dobře odpovídá přednáškám z~kvantové chemie. 
\item Peter W. Atkins, Ronald S. Friedman, \textit{Molecular Quantum Mechanics} Oxford University Press, 2010. Tato učebnice navazuje na Atkinsův základní kurz fyzikální chemie. Kromě kvantové chemie se čtenář seznámí i s~jinými aspekty molekulární kvantové teorie, kupříkladu s~teoretickou spektroskopií.  
\item John P. Lowe, Kirk A. Peterson, \textit{Quantum Chemistry} Elsevier, 2006. Zajímavá učebnice, vhodná zejména pro ty ze čtenářů, kteří si rádi probírané koncepty sami vyzkouší. Autor se hojně věnuje H\"uckelově teorii, se kterou dokáže překvapivě mnoho.
\item Jean-Pierre Launay, Michel Verdaguer, \textit{Electrons in Molecules. From Basic Principle to Molecular Electronics} Oxford University Press, 2014. Kvantová chemie v~moderním kontextu vědy o~materiálech. 
\item Rudolf Polák a Rudolf Zahradník, \textit{Kvantová chemie. Základy teorie a aplikace} SNTL, 1985. Klasické česky psané dílo, které se i po letech dobře čte, mimo jiné i díky svižnému slohu.
\item Jiří Fišer, \textit{Úvod do kvantové teorie}. Academia, 1983. Kniha vhodná zejména pro ty ze čtenářů, kteří by se rádi seznámili s~metodami lineární algebry v~kvantové chemii.
\item Lubomír Skála, \textit{Kvantová teorie molekul} Univerzita Karlova, 1995. Rozsahem nevelká skripta obsahují detailní a srozumitelný popis základních kvantově-chemických přístupů.

\end{itemize}
