Do této chvíle jsme se zabývali pouze pohybem jedné částice v~jednom rozměru. Tušíme přitom, že popis pohybu dvou a více částic (ve dvou a více rozměrech) bude o~něco složitější. Pokud nás ale zajímají jen dvě částice, zůstane vše poměrně jednoduché. Řešení problému dvou částic je přitom v~chemii velmi důležité. Budeme ho potřebovat při hledání energetických stavů rotující či vibrující molekuly nebo při popisu atomů vodíkového typu.

\subsection{Pohyb nezávislých částic: metoda separace proměnných}

 Uvažujme pohyb dvou částic podél osy \textit{x}, přičemž předpokládáme, že částice na sebe vzájemně nepůsobí. Hamiltonián pak můžeme zapsat jako součet Hamiltoniánů popisujících na jednotlivé částice:

\begin{equation}
\hat{H}=\hat{H_{1}}+\hat{H_{2}},
\label{rov:2č-nez1}
\end{equation}

\noindent kde $\hat{H_{1}}$ závisí toliko na souřadnici první částice $x_{1}$ a $\hat{H_{2}}$ pouze na souřadnici druhé částice $x_{2}$. Schr\"odingerova rovnice pak bude vypadat následovně:

\begin{equation}
(\hat{H_{1}}+\hat{H_{2}})\psi(x_{1},x_{2})=E\psi(x_{1},x_{2})
\label{rov:2č-nez2}
\end{equation}

Vlnovou funkci pro dvě částice budeme hledat ve tvaru součinu vlnových funkcí popisujících nezávislé částice:

\begin{equation}
\psi(x_{1},x_{2})=\psi_{1}(x_{1})\psi_{2}(x_{2})
\label{rov:2č-nez3}
\end{equation}

Takovémuto zápisu říkáme separace proměnných. Nejspíše je zřejmé, proč hledáme řešení v~tomto tvaru. Vlnová funkce je spojena s~pravděpodobností výskytu elektronu. Čtverec vlnové funkce na levé straně rovnice \ref{rov:2č-nez3} nám tak udává pravděpodobnost, že se částice 1 nachází v~poloze $x_{1}$ a zároveň se částice 2 nachází v~poloze $x_{2}$. Pravou stranou rovnice \ref{rov:2č-nez3} pak tvrdíme, že tato pravděpodobnost je rovna pravděpodobnosti, že se částice 1 nachází v~poloze  $x_{1}$ nezávisle na poloze částice 2, násobena pravděpodobností nalezení částice 2 v~bodě $x_{2}$ bez ohledu na polohu částice 1. Jinými slovy předpokládáme, že oba jevy jsou nezávislé. To je předpoklad rozumný, neboť obě částice na sobě silově nepůsobí. Znovu zdůrazněme, že vlnová funkce $\psi_{1}$ závisí pouze na souřadnicích částice 1 a vlnová funkce $\psi_{1}$ na souřadnicích částice 2. Nyní přepišme Schr\"odingerovu rovnici pomocí vlnové funkce v~separovaném tvaru:

\begin{equation}
\hat{H_{1}}\psi_{1}(x_{1})\psi_{2}(x_{2})+\hat{H_{2}}\psi_{1}(x_{1})\psi_{2}(x_{2})=E\psi_{1}(x_{1})\psi_{2}(x_{2})
\label{rov:2č-nez44}
\end{equation}


Hamiltonián první částice působí pouze na první částici. Vlnovou funkci druhé částice tedy můžeme brát jako konstantu. Po vydělení $\psi_{1}\psi_{2}$ upravíme Schr\"odingerovu rovnici na následující tvar: 

\begin{equation}
\frac{\hat{H_{1}}\psi_{1}(x_{1})}{\psi_{1}(x_{1})}+\frac{\hat{H_{2}}\psi_{2}(x_{2})}{\psi_{2}(x_{2})}=E
\label{rov:2č-nez5}
\end{equation}

První ze sčítanců na levé straně rovnice \ref{rov:2č-nez5} je funkcí pouze souřadnice $x_{1}$, druhý pak pouze souřadnice $x_{2}$. Součet obou členů musí být přitom konstantní pro každou kombinaci  $x_{1}$ a $x_{2}$. To obecně není možno splnit jinak než, že oba dva sčítance jsou rovny konstantám, které si označíme jako $E_{1}$ a $E_{1}$:

\begin{equation}
\hat{H_{1}}\psi_{1}(x_{1})=E_{1}\psi_{1}(x_{1})
\label{rov:2č-nez6}
\end{equation}
\begin{equation}
\hat{H_{2}}\psi_{2}(x_{2})=E_{2}\psi_{2}(x_{2})
\label{rov:2č-nez7}
\end{equation}
\begin{equation}
E=E_{1}+E_{2}
\label{rov:2č-nez8}
\end{equation}

Schr\"odingerova rovnice pro dvě proměnné se nám tak rozpadla na dvě Schr\"odingerovy rovnice o~jedné proměnné. Místo jedné složité úlohy tak vyřešíme dvě úlohy jednoduché. Energie pohybu obou částic dohromady je dána součtem energií jednotlivých částic a vlnová funkce pro dvě částice je dána součinem dvou jednočásticových vlnových funkcí. Naše úvahy můžeme snadno rozšířit na situaci, kdy se
\begin{itemize}
\item jedna jedna částice pohybuje ve dvou rozměrech: 

\begin{equation}
\psi(x,y)=\psi_{1}(x)\psi_{2}(y)
\label{rov:2č-nez9}
\end{equation}

\item dvě částice pohybují ve třech rozměrech:

\begin{equation}
\psi(x_{1},y_{1},z_{1},x_{2},y_{2},z_{2})=\psi_{1}(x_{1},y_{1},z_{1})\psi_{2}(x_{2},y_{2},z_{2})
\label{rov:2č-nez4}
\end{equation}

\item N částic pohybuje ve třech rozměrech:
\begin{equation}
\psi(x_{1},y_{1},z_{1},...,x_{i},y_{i},z_{i},...,x_{N},y_{N},z_{N})=
\prod_{i=1}^N\psi_{i}(x_{i},y_{i},z_{i})
\label{rov:2č-Nčástic}
\end{equation}

\end{itemize}

Transformace mnohačásticové (mnohadimenzionální) Schr\"odingerovy rovnice na řadu jednočásticových Schr\"odingerových rovnic patří k~základním postupům kvantové teorie molekul. V~případě pohybu nezávislých částic neprovádíme žádné přiblížení. Technika separace proměnných se ale používá i v~případech, kdy pohyb částic striktně nezávislý není.     


{\color{red}XXXXXXXXXXXXXXXX ČÁSTICE VE 2D JÁMĚ XXXXXXXXXXXXXXXXXXXXX}

\subsection{Dvě interagující částice}


Podívejme se nyní na pohyb dvou částic, které na sebe silově působí, například proton a elektron. Potenciální energie je funkcí souřadnic obou částic. Protože na sebe částice silově působí, nemůžeme nyní bezmyšlenkovitě použít metodu separace proměnných jako v~předchozím odstavci. Po určitém úsilí se nám to však přesto podaří. Klíčem bude transformace souřadnic. Použijeme-li kartézské souřadnice, potřebujeme dohromady tři souřadnice pro částici 1 a tři souřadnice pro částici 2. Energetické působení mezi částicemi ale závisí pouze na relativní poloze obou částic. Proto si místo souřadnic $(x_{1},y_{1},z_{1},x_{2},y_{2},z_{2})$ vycházejících z~počátku soustavy souřadnic zavedeme relativní souřadnice $(x,y,z)$:

\begin{eqnarray*}
x=x_{1}-x_{2}\\
y=y_{1}-y_{2}\\
z=z_{1}-z_{2}\\
\end{eqnarray*}
\noindent či ve vektorovém zápisu
\begin{eqnarray*}
\textbf{r}=\textbf{r}_{1}-\textbf{r}_{2}
\end{eqnarray*}

Potenciální energie již není funkcí šesti souřadnic kartézských, ale pouze tří souřadnic relativních.
 
Zbavili jsme se tak kartézských souřadnic jednotlivých částic při popisu potenciální energie. Nyní se ale musíme vypořádat ještě s~energií kinetickou. Kromě relativních souřadnic \textbf{r} zavedeme ještě souřadnice těžiště \textbf{R} 

\begin{equation}
\textbf{R}=\frac{m_{1}\textbf{r}_{1}+m_{2}\textbf{r}_{2}}{m_{1}+m_{2}}.
\label{rov:2č-těžiště}
\end{equation}

Pohyb obou částic pak můžeme vyjádřit pomocí pohybu těžiště a relativního pohybu obou částic. Kartézské souřadnice $\textbf{r}_{1}$ a $\textbf{r}_{2}$ si nyní můžeme vyjádřit pomocí souřadnic těžiště \textbf{R} a vektoru relativní polohy \textbf{r}:

\begin{eqnarray}
\textbf{r}_{1}=\textbf{R}+\frac{m_{2}}{m_{1}+m_{2}}\textbf{r}
\label{rov:2č-r1}\\
\textbf{r}_{2}=\textbf{R}-\frac{m_{1}}{m_{1}+m_{2}}\textbf{r}
\label{rov:2č-r2}
\end{eqnarray}

Kinetická energie systému dvou částic je v~klasické fyzice definována jako

\begin{equation}
T=\frac{1}{2}m_{1}|\dot{\textbf{r}}_{1}|^{2}+\frac{1}{2}m_{2}|\dot{\textbf{r}}_{2}|^{2}
\end{equation}

Pomocí vztahů \ref{rov:2č-r1} a \ref{rov:2č-r2} si můžeme vztah pro kinetickou energii upravit na 

\begin{equation}
T=\frac{1}{2}M|\dot{\textbf{R}}|^{2}+\frac{1}{2}\mu| \dot{\textbf{r}}|^{2},
\label{rov:2č-Tmmu}
\end{equation}

kde $ M=m_{1}+m_{2} $ je celková a $ \mu $ redukovaná hmotnost definovaná jako

\begin{equation}
\mu=\frac{m_{1}m_{2}}{m_{1}+m_{2}}.
\end{equation}

Vztah (\ref{rov:2č-Tmmu}) pro kinetickou energii můžeme přepsat pomocí hybností

 \begin{equation}
T=\frac{|\textbf{p}_{M}|^{2}}{2M}+\frac{|\textbf{p}_{\mu}|^{2}}{2\mu},
\label{rov:2č-Thybnosti}
\end{equation}

kde jsme si zavedli vektory hybnosti spojené s~pohybem molekuly jako celku ($ \textbf{p}_{M} $) a s~relativním pohybem obou částic ($ \textbf{p}_{\mu} $). 

Uvažujme nyní, že potenciální energie je pouze funkcí vzdálenosti obou částic (mluvíme o~tzv. centrálním potenciálu)

\begin{equation}
V_{\mu}=V(r).
\end{equation}

Celkovou energii je pak možné zapsat jako součet členů závisejících pouze na souřadnicích a hybnostech těžiště a souřadnicích a hybnostech relativního pohybu obou částic. Při přechodu do kvantové mechaniky dostaneme pro Hamiltonián:

\begin{equation}
\hat{H}=\frac{{\hat{p}}_{M}^{2}}{2M}+\frac{{\hat{p}}_{\mu}^{2}}{2\mu}+\hat{V}_{\mu}
\end{equation}

První člen popisuje translační kinetickou energii, tzn. obě částice urazí stejnou dráhu a vzdálenosti mezi nimi se nemění. Naopak druhý a třetí člen závisejí pouze na vzdálenosti obou částic. Získali jsme Hamiltonián, který nám umožňuje provést separaci proměnných, tj. vlnovou funkci nyní můžeme rozdělit část popisující translaci a část popisující relativní pohyb. 

\begin{equation}
\psi=\psi_{M}\psi_{\mu}
\end{equation}

Schr\"odingerova rovnice se tak rozpadne na  rovnice dvě:

\begin{eqnarray}
\frac{{\hat{p}}_{M}^{2}}{2M}\psi_{M}=E_{M}\psi_{M}
\label{rov:2č-schr-trans}
\\
\left(\frac{{\hat{p}}_{\mu}^{2}}{2\mu}+\hat{V}_{\mu}\right)\psi_{\mu}=E_{\mu}\psi_{\mu}
\label{rov:2č-schr-vibr}
\end{eqnarray}

a stejně tak energii můžeme rozdělit na příspěvky energie translační a vnitřní (relativní) $ E=E_{M}+E_{\mu} $. Translační pohyb prozatím ponechme stranou a podívejme se na Schr\"odingerovu rovnici popisující relativní pohyb dvou částic. Začněme zápisem Hamiltoniánu v~kartézských souřadnicích:

\begin{equation}
\hat{H}_{\mu}=\frac{{\hat{p}}_{\mu}^{2}}{2\mu}+\hat{V}_{\mu}=\left(\frac{-\hbar^{2}}{2\mu}\right)\nabla^{2}+\hat{V}_{\mu}=\left(\frac{-\hbar^{2}}{2\mu}\right)\left(\frac{\partial^{2}}{\partial x^{2}}+\frac{\partial^{2}}{\partial y^{2}}+\frac{\partial^{2}}{\partial z^{2}}\right)+\hat{V}_{\mu}.
\label{rov:2č-SferHamil}
\end{equation}

\noindent kde \textit{x,y} a \textit{z} jsou kartézské složky relativních souřadnic. Pokud máme v~úmyslu zabývat se pohybem v~centrálním poli, není kartézský souřadnicový systém příliš vhodný. Centrální pole se vyznačuje sféricky symetrickým potenciálem, proto je přirozené přejít ke sférickým souřadnicím. {\color{red}V kapitole XXX} jsme si odvodili vztah pro Laplacián ve sférických souřadnicích, který nyní využijeme:

\begin{equation}
\nabla^{2}=\frac{\partial^{2}}{\partial r^{2}}+\frac{2}{r}\frac{\partial}{\partial r}+\frac{1}{r^{2}}\frac{\partial^{2}}{\partial \theta^{2}}+\frac{1}{r^{2}}\cot\theta\frac{\partial}{\partial \theta}+\frac{1}{r^2\sin^2\theta}\frac{\partial^{2}}{\partial \phi^{2}}=\frac{\partial^{2}}{\partial r^{2}}+\frac{2}{r}\frac{\partial}{\partial r}-\frac{1}{r^2\hbar^2}\hat{L}^2
\label{rov:2č-Momenthybnosti}
\end{equation}

K~zápisu Hamiltoniánu ve sférických souřadnicích jsme tedy použili operátor momentu hybnosti. {\color{red}Z kapitoly XXX} víme, že vlastní funkce $\hat{L}^2$ jsou sférické harmonické funkce. Hamiltonův operátor je zapsán ve tvaru, který umožňuje provést separaci proměnných. 


Můžeme si nyní položit otázku, zda Hamiltonián komutuje s~$\hat{L}^2$, tj. zda platí

\begin{equation}
[\hat{H},\hat{L}^{2}]=[\hat{T},\hat{L}^{2}]+[\hat{H},\hat{V}^{2}]=0   
\label{rov:2č-komutator}
\end{equation}

Člen s~operátorem kinetické energie se po dosazení z~rovnice \ref{rov:2č-Momenthybnosti} rozpadne na dva členy: 

\begin{equation}
\left[\hat{T},\hat{L}^{2}\right]=\left[-\frac{\hbar^{2}}{2m}\left(\frac{\partial^{2}}{\partial r^{2}}+\frac{2}{r}\frac{\partial}{\partial r}\right)+\frac{1}{2mr^{2}}\hat{L}^2,\hat{L}^2\right]=-\frac{\hbar^2}{2m}\left[\frac{\partial^{2}}{\partial r^{2}}+\frac{2}{r}\frac{\partial}{\partial r},\hat{L}^2\right]+\frac{1}{2m}\left[\frac{1}{r^2}\hat{L}^2,\hat{L}^2\right],
\label{rov:2č-komutatorT}
\end{equation}
z~nichž první člen je nulový, protože operátor $\hat{L}^{2}$ nepůsobí na  $ r $, a druhý člen je nulový, protože $ \hat{L}^2 $ komutuje sám se sebou. Podobně budeme postupovat u~komutátoru s~operátorem potenciální energie. Potenciální energie u~centrálního pole závisí pouze na $r$, zatímco $\hat{L}^2$ závisí na $ \phi $ a $ \theta $, proto bude i tento komutátor nulový.  Tím jsme dokázali, že Hamiltonián u~systému s~centrálním polem komutuje s~$\hat{L}^2$. Podobně bychom dokázali, že
\begin{equation}
[\hat{H},L_{z}]=0.
\end{equation}
 
Hamiltonián tedy komutuje jak se čtvercem operátoru momentu hybnosti, tak s~jeho z-tovou složkou. To především znamená, že při pohybu dvou částic v~centrálním potenciálu se zachovává hybnost i její z-tová složka. 


Tvar Hamiltoniánu \ref{rov:2č-SferHamil} nám umožňuje zapsat vlnovou funkci v~polárních souřadnicích ve tvaru součinu sférické harmonické funkce a nějaké doposud neurčené radiální vlnové funkce:

\begin{equation}
\psi=R(r)Y^m_l(\theta,\psi)
\end{equation}

Takto definovanou vlnovou funkci nyní dosadíme do Schr\"odingerovy rovnice, kde je Hamiltonián vyjádřen podle rovnic \ref{rov:2č-SferHamil} a \ref{rov:2č-Momenthybnosti}.

\begin{equation}
\begin{split}
\hat{H}\psi= & -\frac{\hbar^2}{2\mu}\left(\frac{\partial}{\partial r^2} +\frac{\hbar^2}{2\mu}\frac{2}{r}\right)R(r)Y^m_l(\theta,\psi) + \\
& +\frac{\hbar^2}{2\mu}\frac{1}{r^2\hbar^2}\hat{L}^2R(r)Y^m_l(\theta,\psi)+\hat{V}R(r)Y^m_l(\theta,\psi)=ER(r)Y^m_l(\theta,\psi).
\end{split}
\end{equation}

Po vydělení $ Y^m_l(\theta,\psi) $ a aplikaci $ \hat{L}^2 $ posléze dostaneme tzv. radiální Schr\"odingerovu rovnici:

\begin{equation}
-\frac{\hbar^2}{2\mu}\left(R''+\frac{2}{r}R'\right)+\frac{l\left(l+1\right)\hbar^2}{2\mu r^2}R+\hat{V}R=ER.
\end{equation}













