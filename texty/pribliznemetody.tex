V~minulých oddílech jsme si ukázali řešení Schr\"{o}dingerovy rovnice pro některé jednoduché případy. Analyticky řešitelných úloh je ale velmi málo. Obecně nejsme schopni (ani v~klasické mechanice) řešit úlohy s~více než dvěma částicemi. Jsme pak odkázáni na řešení přibližná. V~této kapitole si představíme nejdůležitější přibližné přístupy, se kterými se v~kvantové chemii setkáváme.

\subsection{Variační princip}

Přírodní zákony mohou být často formulovány různými způsoby. Principy klasické mechaniky lze například formulovat pomocí soustavy diferenciálních rovnic (např. Newtonových), ale stejně platný je tzv. princip minimálního účinku. Podobně zákony klasické optiky můžeme formulovat pomocí zákonů odrazu a lomu, ale stejně dobře pomocí tzv. Fermatova principu, dle kterého se paprsek mezi dvěma body pohybuje po dráze, na které stráví nejkratší čas. Fermatův princip je příkladem variačního principu. S~variačním principem se hojně setkáváme také v~kvantové mechanice. 

\subsubsection{Variační princip v~kvantové mechanice}

Dle variační teorému je vlastní hodnota Hamiltoniánu pro jakoukoli \textbf{přibližnou normovanou vlnovou funkci} větší nebo rovna energii základního stavu systému

\begin{equation}
\boxed{\int\phi^*\hat{H}\phi d\tau\geq E_0,}
\label{rov:aprox-variacniteorem}
\end{equation}

\noindent kde $ \phi $ je přibližná vlnová funkce a $ E_0 $ energie základního stavu. Toto tvrzení je nesmírně užitečné. Představme si, že jsme nějakým způsobem odhadli přibližné řešení Schr\"{o}dingerovy rovnice. Variační princip nám umožní rozhodnout, zda toto naše řešení je lepší nebo horší než nějaké jiné přibližné řešení: správnější bude řešení s~nižší energií. Jak uvidíme níže, variační princip nám navíc umožňuje přibližné řešení efektivně nalézt. 


 Ukažme si nyní, že variační princip je ekvivalentní se Schr\"{o}dingerovou rovnicí. Neznámou vlnovou funkci můžeme zapsat ve tvaru lineární kombinace vlastních funkcí Hamiltonova operátoru

\begin{equation}
\phi=\sum_k c_k \psi_k. 
\label{rov:aprox-rozvojbaze}
\end{equation}

\noindent Připomeňme si zde, že vlastní funkce Hamiltonova operátoru vytváří úplný soubor funkcí, takže výše uvedený rozvoj můžeme určitě udělat. Rozvoj nyní dosadíme do levé strany rovnice \ref{rov:aprox-variacniteorem}
\begin{equation}
\int\phi^*\hat{H}\phi d\tau=\int\left( \sum_k c_k^* \psi_k^*\hat{H}\sum_l c_l\psi_l\right)d\tau =\sum_k \sum_l c_k^*c_l\int\psi_k^*\hat{H}\psi_l d\tau=\sum_k \sum_l c_k^*c_lE_k\delta_{kl}.
\end{equation}

\noindent Rovnici si nyní můžeme přepsat do jednoduššího tvaru
\begin{equation}
\int\phi^*\hat{H}\phi d\tau=\sum_k |c_k|^2E_k.
\label{rov:aprox-vari2}
\end{equation}

\noindent Protože je naše funkce $ \phi $ normovaná, platí $\sum_k |c_k|^2=1 $. Z~definice přitom platí, že $E_0 \leq E_1 \leq E_2$ atd. Tudíž výraz \ref{rov:aprox-vari2} bude určitě větší než $ E_0 $, což není nic jiného než variační teorém.

Při praktickém použití uvažujeme většinou  přibližnou vlnovou funkci závisející na několika parametrech. Minimalizace variačního integrálu

\begin{equation}
\epsilon=\frac{\int\phi^*\hat{H}\phi d\tau}{\int\phi^*\phi d\tau}\geq E_0
\label{rov:aprox-variacniintegrál}
\end{equation}

\noindent vůči těmto parametrům pak umožní nalezení jejich optimálních hodnot. Jmenovatel ve výše uvedeném výrazu je samozřejmě pro normované funkce $ \phi $ jednotkový.

\begin{priklad}
Platnost variačního principu si ukážeme na příkladu jednorozměrné potenciálové jámy. Analytické řečení této úlohy vede k~hodnotám energie $ E_n=\frac{n^2h^2}{8mL^2}$. Přibližná vlnová funkce musí splňovat okrajové podmínky, tedy $ \phi(0)=0 $ a $ \phi(L)=0 $. Nejjednodušší takovou funkcí bude nejspíš kvadratická funkce. Nechť má tedy přibližná funkce $\phi$ následující tvar
 
\begin{displaymath} 
\phi=x(L-x).
\end{displaymath}
 
Funkce $\phi$ není normovaná, proto nejprve určíme normu
 
\begin{displaymath} 
N=\int_0^L \phi^*\phi dx=\int_0^L (x(L-x))^2 dx=\frac{L^5}{30}.
\end{displaymath}
 
\noindent Normovaná přibližná funkce tedy vypadá následovně,
 
\begin{displaymath} 
\sqrt{\frac{30}{L^5}} x(L-x).
\end{displaymath}
 
Připomeňme si ještě, jak vypadá hamiltonián pro částici v~jednodimenzionální potenciálové jámě
 
\begin{displaymath} 
\hat{H}=\frac{-\hbar^2}{2m}\frac{d^2}{d x^2}.
\end{displaymath}
 
Nyní můžeme uplatnit vztah \ref{rov:aprox-variacniteorem}

   \begin{displaymath} 
E_k=\int\phi^*\hat{H}\phi dx=\frac{-\hbar^2}{2m}\frac{30}{L^5}\int_0^L x(L-x)\left(\frac{d^2 [x(L-x)]}{d x^2}\right)dx
\end{displaymath}
 
 Po vyčíslení dostaneme
 
    \begin{displaymath} 
E_k=\frac{5h^2}{4\pi^2 mL^2}\geq \frac{h^2}{8mL^2}=E_0.    
\end{displaymath}
 
Na konkrétním příkladu jsme si tak ukázali platnost variačního teorému. 
\end{priklad}
 
\subsubsection{Lineární variační funkcionál}
\label{kap:Lin Var funkcional}
 
 V~kvantové chemii je užitečné hledat řešení jako lineární kombinaci předem definovaných funkcí a optimalizovat pouze příspěvky jednotlivých bázových funkcí k~celkové vlnové funkci, tj. optimalizovat tzv. rozvojové koeficienty. Tak v~chemii jsme si kupříkladu navykli hledat molekulové orbitaly jako lineární kombinace atomových orbitalů. Zapišme tedy vlnovou funkci pomocí lineární kombinace bázových funkcí
 
 \begin{equation}
 \phi=\sum_k c_k f_k,
 \label{rov:aprox-Rayleigh-Ritz} 
 \end{equation}
 
\noindent kde $ f_k $ je sada nám známých vhodných funkcí a $ c_k $ jsou rozvojové koeficienty. Ty budeme variačně optimalizovat. Takto definovanou vlnovou funkci nyní dosadíme do výrazu variačního integrálu, \ref{rov:aprox-variacniintegrál}
 
 \begin{equation}
\epsilon = \frac{\int \phi^{\ast}\hat{H}\phi \mathrm{d} \tau}{\int \phi^{\ast} \phi \mathrm{d} \tau}=\frac{\int\sum_k c_k f_k\hat{H}\sum_l c_l f_l d\tau}{\int\sum_k c_k f_k\sum_l c_l f_l d\tau}=\frac{\sum_{k,l} c_k c_l \int f_k \hat{H} f_ld\tau}{\sum_{k,l} c_k c_l \int f_k f_ld\tau}.
\label{rov:aprox-variacniintegrál2}
\end{equation}
 
 \noindent Definujme si nyní matici překryvu $\mathbf{S}$ pomocí jejich maticových elementů
 \begin{equation}
 S_{kl}=\int f_k f_ld\tau
\end{equation}  
 
\noindent  a Hamiltonovu matici $\mathbf{H}$ s elementy
 
  \begin{equation}
 H_{kl}=\int f_k \hat{H} f_ld\tau.
\end{equation}  
 
 \noindent Variační integrál pak nabude tvaru
 
   \begin{equation}
 \epsilon=\frac{\sum_{k,l}c_k c_l H_{kl}}{\sum_{k,l}c_k c_l S_{kl}}.
\end{equation}  
 
 \noindent Nejlepší řešení dostaneme pro funkci, pro který bude variační integrál minimální. To vede k~podmínce
 \begin{equation}
 \frac{\partial \epsilon}{\partial c_i}=0.
 \label{rov:aprox-derivace} 
 \end{equation}
 
 \noindent Derivaci čitatele si rozepíšeme následovně
 
 \begin{equation}
 \frac{\partial}{\partial c_i} \sum_{k,l}c_k c_l H_{kl}=\sum_{k,l} \frac{\partial}{\partial c_i} (c_k c_l H_{kl})=\sum_{k,l} \left(\frac{\partial c_k}{\partial c_i}c_l H_{kl} + \frac{\partial c_l}{\partial c_i}c_k H_{kl} \right)
   \end{equation}
 
\noindent  Derivace jednotlivých členů se dále zjednoduší, uvažujeme-li
  \begin{equation}
 \frac{\partial c_k}{\partial c_i}=\delta_{ki}.
   \end{equation}
 
 \noindent Dostaneme tak konečný výraz pro derivaci čitatele z~rovnice \ref{rov:aprox-derivace}
 
  \begin{equation}
 \frac{\partial}{\partial c_i} \sum_{k,l}c_k c_l H_{kl}=\sum_{k} c_k H_{ki}+\sum_{l} c_l H_{li}.
   \end{equation}
 
 \noindent Použitím posledního vztahu a uplatněním vztahu pro derivaci podílu nakonec dostaneme upravený tvar rovnice \ref{rov:aprox-derivace}
 
  \begin{equation}
 \frac{\partial\epsilon}{\partial c_i}=\frac{\sum_{k} c_k H_{ki}+\sum_{l} c_l H_{li}}{\sum_{k,l}c_kc_lS_{kl}}-\frac{\left(\sum_{k} c_k S_{ki}+\sum_{l} c_l S_{li}\right)\sum_{k,l}c_k c_l H_{kl}}{\left(\sum_{k,l}c_k c_l\right)^2}
   \end{equation}
 
 \noindent Protože hledáme minimum funkce, položíme variační integrál roven nule (použili jsme znovu definice variačního integrálu \ref{rov:aprox-variacniintegrál2})
 
  \begin{equation}
 \frac{\partial\epsilon}{\partial c_i}=\frac{\sum_k c_{k}\left(H_{ik}-\epsilon S_{ik}\right)+\sum_l c_{l}\left(H_{li}-\epsilon S_{li}\right)}{\sum_{k,l}c_k c_l}=0
     \end{equation}
 Jelikož na označení sčítacího indexu v~sumě nezáleží, změní se poslední výraz na 
\begin{equation} 
 \frac{\partial\epsilon}{\partial c_i}=\frac{2\sum_k c_{k}\left(H_{ik}-\epsilon S_{ik}\right)}{\sum_{k,l}c_k c_l}=0
\end{equation}
 
\noindent Derivace bude nulová, pokud je nulový čitatel. Tím dostáváme soustavu tzv. sekulárních rovnic

  \begin{equation}
 \sum_k c_{k}\left(H_{ik}-\epsilon S_{ik}\right)=0
     \end{equation}
 
Jde o~soustavu lineárních rovnic, jejímž řešením jsou rozvojové koeficienty $c_k$. To ovšem za předpokladu, že známe $\epsilon$. To je ale hodnota, kterou hledáme! Všimněme si, že jedno řešení známe okamžitě $c_k=0$. Jde o~tzv. triviální řešení. To nás ale nezajímá, nulové řešení totiž nepředstavuje vlnovou funkci. Lineární algebra nás učí, že soustava rovnic s~nulovou pravou stranou má netriviální řešení pouze tehdy, jestliže je determinant soustavy (v~tomto případě sekulární determinant) nulový
 
   \begin{equation}
 |H_{ik}-\epsilon S_{ik}|=0.
     \end{equation}
 
Sekulární determinant představuje algebraickou rovnici pro neznámou hodnotu $\epsilon$. Pro rozvoj  vlnové funkce do dvou bázových funkcí půjde o~rovnici kvadratickou, pro rozsáhlejší rozvoje půjde o~rovnice vyšších řádů.

\begin{priklad}
Jednodimenzionální potenciálovou jámu si nyní vyřešíme i pomocí bázového rozvoje \ref{rov:aprox-Rayleigh-Ritz}. Vlnovou funkci definujeme ve tvaru $ \phi=c_1x^2(L-x)+c_2x(L-x)^2 $. Postup je následující:

\begin{itemize}

\item Spočítáme všechny integrály $H_{k,l}$ a $S_{k,l}$
\item Položíme sekulární determinant roven nule a vypočítáme energie $\epsilon$
\item Pro každou hodnotu $\epsilon$ vypočítáme rozvojové koeficienty ze sekulárních rovnic

\end{itemize}

Příklad vyřešený v~programu Mathematica je dostupný na webových stránkách předmětu.

\end{priklad} 
 
\begin{priklad}
\label{priklad var princip}
Použití variačního principu si ukážeme na zápisu molekulového orbitalu homodiatomika (např. H$_{2}$) jako lineární kombinace dvou atomových orbitalů (A~a B) ležících na jednotlivých atomech
\begin{displaymath} 
\phi = c_A\chi_A+c_B\chi_B.    
\end{displaymath}
Nyní si můžeme vyčíslit maticové elementry $H_{kl}$ a $S_{kl}$

\begin{eqnarray}
H_{AB}&=&\int \chi_A\hat{H}\chi_Bd\tau\nonumber\\
H_{AA}&=&\int \chi_A\hat{H}\chi_Ad\tau=H_{BB}\nonumber\\
S_{AB}&=&\int \chi_A \chi_B d\nonumber\tau
\end{eqnarray}   

Dalším krokem bude sestavení sekulárního determinantu
 \begin{displaymath}
    \left|\begin{array}{cc@{\quad}r}
     H_{AA} -\epsilon & H_{AB}-\epsilon S_{AB}  \\
     H_{AB} -\epsilon S_{AB} & H_{BB}-\epsilon  \\
    \end{array}\right |=0.
    \end{displaymath}

Sekulární determinant posléze s využitím $H_{AA} = H_{BB}$ vyřešíme a dostaneme vztah pro $\epsilon$

\begin{displaymath}
\epsilon_{1,2}=\frac{H_{AA}\pm H_{AB}}{1\pm S_{AB}}.
\end{displaymath}

Dále vyřešíme \textbf{sekulární rovnice}
\begin{eqnarray}
c_A(H_{AA}-\epsilon S_{AA})+c_B(H_{AB}-\epsilon S_{AB})&=&
c_A(H_{AA}-\epsilon)+c_B(H_{AB}-\epsilon S_{AB})=0\nonumber\\
c_A(H_{BA}-\epsilon S_{BA})+c_B(H_{BB}-\epsilon S_{BB})&=&
c_A(H_{AB}-\epsilon S_{AB})+c_B(H_{AA}-\epsilon)=0.\nonumber
\end{eqnarray}

Do sekulárních rovnic si dosadíme vztahy za $ \epsilon_{1,2} $
a po vyčíslení dostaneme pro  $c_A=c_B$ (pro $ \epsilon_1 $) a $c_A=-c_B$ (pro $ \epsilon_2 $). Ze dvou atomových orbitalů tedy dostaneme vazebný molekulový orbital $ \phi_1 $
\begin{displaymath}
\phi_1=c_A(\chi_A+\chi_B)
\end{displaymath}
a antivazebný molekulový orbital $ \phi_2 $
\begin{displaymath}
\phi_2=c_A(\chi_A-\chi_B).
\end{displaymath} \vspace{-0.9cm}
\end{priklad}
 
\subsection{Poruchová teorie}
\label{kap:PoruchovaTeorie}
 
Předpokládejme, že řešíme Schr\"odingerovy rovnici pro nějaký složitý problém

\begin{equation}
\hat{H}\psi_n=E_n \psi_n,
\end{equation}


\noindent Známe přitom řešení jednoduššího problému, který se od původního problému moc neliší
 
\begin{equation}
\hat{H}^{(0)}\psi_n^{(0)}=E_n^{(0)} \psi_n^{(0)}.
\end{equation}
 
\noindent Jinými slovy, od jednoduchého problému přejdeme k~problému složitějšímu přidáním malé poruchy. K~řešení problémů tohoto typu používáme poruchovou teorii.
 
\begin{priklad} 
Chceme popsat vibraci molekuly v~anharmonickém potenciálu s~kubickým a kvartickým členem
  
\begin{equation}
\hat{H}=-\frac{\hbar^2}{2m}\frac{d^2}{dx^2}+\frac{1}{2}kx^2+cx^3+dx^4
\nonumber
\end{equation}
 
Přičemž jsme schopni vyřešit problém pro harmonický potenciál
 
\begin{equation}
\hat{H}^{(0)}=-\frac{\hbar^2}{2m}\frac{d^2}{dx^2}+\frac{1}{2}kx^2.
\nonumber
\end{equation}
 
Hamiltonián harmonického systému tak representuje neporušený systém a kubické a kvartické členy jsou malá porucha
 
\begin{equation}
\hat{H}=\hat{H}^{(0)}+cx^3+dx^4.
\end{equation}

Doufáme přitom, že se řešení pro systém s~neporušený systém nebude dramaticky lišit od plného řešení, které získáme malou korekcí neporušeného řešení.
\end{priklad}

Obecně si tedy Hamiltonián definujeme pomocí přibližného Hamiltoniánu a poruchy

\begin{equation}
\hat{H}=\hat{H}^{(0)}+\lambda\hat{H}',
\label{rov:approx:ham}
\end{equation}

\noindent přičemž $ \lambda $ je parametr sloužící k~postupnému zapnutí poruchy ($ \lambda = 0 $ znamená, že máme neporušený systém, $ \lambda=1 $ je naopak plně zapnutá porucha).

Protože Hamiltonián závisí na poruchovém parametru $\lambda$, bude na poruše záviset také vlnová funkce a energie. Vlnovou funkci a energii si rozepíšeme ve formě Taylorova rozvoje okolo poruchového parametru

\begin{equation}
\psi_{n}=\psi^{(0)}_{n}+\lambda \psi^{(1)}_{n}+\lambda^2 \psi^{(2)}_{n}+...
\label{rov:aprox:poruchawf}
\end{equation}

\begin{equation}
E_{n}=E^{(0)}_{n}+\lambda E^{(1)}_{n}+\lambda^2 E^{(2)}_{n}+...
\end{equation}

V~dalším odvození budeme pracovat s~tím, že vlnová funkce neporušeného systému $\psi^{(0)}_n$ je normalizovaná
\begin{equation}
\int \psi_n^{(0)\ast} \psi_N^{(0)} \mathrm{d} \tau  = \left <\psi^{(0)}_n |\psi^{(0)}_n\right >=1.
\end{equation}
Místo uvažování normalizované vlnové funkce se všemi poruchami budeme uvažovat následující normalizační relaci

\begin{equation}
\int \psi_n^{(0) \ast} \psi_n^{(0)} \mathrm{d} \tau =\left <\psi^{(0)}_n |\psi_n\right >=1.
\label{rov:aprox:IntermediateNormalization}
\end{equation}

\noindent Tímto způsobem nebude celková poruchová vlnová funkce normalizovaná. To ale nevadí, protože vynásobení vlnové funkce konstantou (normou) nemá vliv na hodnotu energie ze Schrödingerovy rovnice $ \hat{H}\psi_n=E_n\psi_n $. Nyní do normalizační podmínky (\ref{rov:aprox:IntermediateNormalization}) dosadíme poruchovou vlnovou funkci \ref{rov:aprox:poruchawf} (pro ušetření místa použijeme braketovou notaci)

\begin{equation}
1=\left <\psi^{(0)}_n |\psi^{(0)}_n \right >+
\lambda \left <\psi^{(0)}_n |\psi^{(1)}_n \right >+
\lambda^2 \left <\psi^{(0)}_n |\psi^{(2)}_n \right > + ...
\end{equation}

První člen pravé strany poslední rovnice je jednotkový, protože vlnová funkce neporušeného systému je normalizovaná. Proto musí být další členy nulové

\begin{eqnarray}
\left <\psi^{(0)}_n |\psi^{(1)}_n \right >=0\\
\left <\psi^{(0)}_n |\psi^{(2)}_n \right >=0. 
\end{eqnarray}

\noindent Nyní si Hamiltonián, energii i vlnovou funkci dosadíme do Schrödingerovy rovnice

\begin{eqnarray}
\lefteqn{ \left(\hat{H}^{(0)}+\lambda \hat{H}'\right)\left(\psi^{(0)}_{n}+\lambda \psi^{(1)}_{n}+\lambda^2 \psi^{(2)}_{n}+...\right)=}
\nonumber\\
&&=\left(E^{(0)}_{n}+\lambda E^{(1)}_{n}+\lambda^2 E^{(2)}_{n}+...\right)\left(\psi^{(0)}_{n}+\lambda \psi^{(1)}_{n}+\lambda^2 \psi^{(2)}_{n}+...\right)
\end{eqnarray}

\noindent Rovnici si přepíšeme do následujícího tvaru

\begin{eqnarray}
\lefteqn{ \lambda^0\left(\hat{H}^{(0)}\psi^{0}_{n}-E^{(0)}_{n}\psi^{(0)}_{n}\right)+
\lambda^1\left(\hat{H}^{(0)}\psi^{(1)}_{n}+\hat{H}'\psi^{(0)}_{n}-E^{(0)}_{n}\psi^{(1)}_{n}-E^{(1)}_{n}\psi^{(0)}_{n}\right)+{}}
\nonumber\\
&& + 
\lambda^2\left(\hat{H}^{(0)}\psi^{(2)}_{n}+\hat{H}'\psi^{(1)}_{n}+E^{(0)}_{n}\psi^{(2)}_{n}-E^{(1)}_{n}\psi^{(1)}_{n}-E^{(2)}_{n}\psi^{(0)}_{n}\right)+...=0.
\end{eqnarray}

\noindent Vzhledem k~tomu, že rovnice musí platit pro libovolnou hodnotu parametru $\lambda$, musí se každá závorka zároveň rovnat nule. Schrödingerova rovnice se tedy rozpadá na následující rovnice

\begin{eqnarray}
\hat{H}^{(0)}\psi^{(0)}_{n} &=&E^{(0)}_{n}\psi^{(0)}_{n}\\
\left(\hat{H}^{(0)}-E^{(0)}_{n}\right)\psi^{(1)}_{n} &=&\left(E^{(1)}_{n}-\hat{H}'\right)\psi^{(0)}_{n}
\label{rov:aprox-lambda1} 
\\
\left(\hat{H}^{(0)}-E^{(0)}_{n}\right)\psi^{(2)}_{n} &=&E^{(2)}_{n}\psi^{(0)}_{n}+\left(E^{(1)}_{n}-\hat{H}'\right)\psi^{(1)}_{n}
\end{eqnarray}

Nyní si odvodíme, jak bude vypadat korekce k~energii z~prvního řádu poruchové metody. Využijeme rovnice \ref{rov:aprox-lambda1} Předpokládejme, že funkce $\psi_n^{(0)}$ tvoří úplný soubor funkcí.
Rovnici \ref{rov:aprox-lambda1} nyní vynásobíme komplexně sdruženou funkcí $ \psi_m^{(0)*} $ a přepíšeme do braketové notace

\begin{equation}
\left < \psi_m^{(0)}|\hat{H}^{(0)}|\psi_n^{(1)} \right >-
E_n^{(0)}\left < \psi_m^{(0)}|\psi_n^{(1)} \right >=
E_n^{(1)}\left < \psi_m^{(0)}|\psi_n^{(0)} \right >-
\left < \psi_m^{(0)}|\hat{H}'|\psi_n^{(0)} \right >.
\label{rov:aprox:poruchaenergie1}
\end{equation}

\noindent První člen v~předchozí rovnici si můžeme upravit s~využitím hermiticity operátoru $\hat{H}^{(0)}$
\begin{equation}
\left < \psi_m^{(0)}|\hat{H}^{(0)}|\psi_n^{(1)} \right >=
\int \psi_m^{(0)*}\hat{H}^{(0)}\psi_n^{(1)}d\tau=
\int \psi_n^{(1)}\hat{H}^{(0)}\psi_m^{(0)*}d\tau=
E_m^{(0)}\left < \psi_m^{(0)}|\psi_n^{(1)} \right >.
\end{equation}

\noindent Rovnice \ref{rov:aprox:poruchaenergie1} pak přejde na tvar
\begin{equation}
\left(E_m^{(0)}-E_n^{(0)}\right)\left < \psi_m^{(0)}|\psi_n^{(1)} \right >=
E_n^{(1)}\left < \psi_m^{(0)}|\psi_n^{(0)} \right >-
\left < \psi_m^{(0)}|\hat{H}'|\psi_n^{(0)} \right >.
\label{rov:aprox:poruchaenergie2}
\end{equation}

\noindent Nyní mohou nastavit dva případy:
\begin{itemize}
\item $m=n$:
Pokud $m=n$, dostaneme okamžitě korekci k~energii


\begin{equation}
\boxed{E_n^{(1)}=\left < \psi_n^{(0)}|\hat{H}'|\psi_n^{(0)} \right >.}
\label{rov:PorusenaEnergie}
\end{equation} 

Porucha energie v~prvním stupni tedy odpovídá poruše zprůměrované přes neporušenou vlnovou funkci.

\item $m \neq n$:
rovnice \ref{rov:aprox:poruchaenergie2} v~tomto případě přejde na tvar
\begin{equation}
\left(E_m^{(0)}-E_n^{(0)}\right)\left < \psi_m^{(0)}|\psi_n^{(1)} \right >=
-\left < \psi_m^{(0)}|\hat{H}'|\psi_n^{(0)} \right >.
\end{equation}
 Poruchu ve vlnové funkci prvního řádu si vyjádříme jako lineární kombinaci bázových funkcí

\begin{equation}
\psi_n^{(1)}=\sum_i c_i \psi^{(0)}_i, 
\end{equation}

dále dostaneme

\begin{equation}
\left(E_m^{(0)}-E_n^{(0)}\right)\sum_i c_i\left < \psi_m^{(0)}| \psi^{(0)}_i \right >=
-\left < \psi_m^{(0)}|\hat{H}'|\psi_n^{(0)} \right >
\end{equation}

a po úpravě 

\begin{equation}
\left(E_m^{(0)}-E_n^{(0)}\right) c_m=
-\left < \psi_m^{(0)}|\hat{H}'|\psi_n^{(0)} \right >.
\end{equation}

Korekci prvního řádu k~vlnové funkci tedy dostaneme pomocí následujícího vztahu 

\begin{equation}
\boxed{\psi_n^{(1)}=\sum_m \frac{\left < \psi_m^{(0)}|\hat{H}'|\psi_n^{(0)} \right >}{E_n^{(0)}-E_m^{(0)}}\psi_m^{(0)}.}
\end{equation} 

\end{itemize}

Podobným způsobem, jaký jsme si zde ukázali pro korekce prvního řádu k~energii a vlnové funkci lze odvodit i vztahy pro vyšší řády poruchové energie. Zde uvedeme pouze konečný vztah pro energii s~poruchou druhého řádu, energie $E_n$ pak bude

\begin{eqnarray}
E_n\approx E_n^{(0)}+E_n^{(1)}+\sum_{m\neq n} \frac{\left | H'_{mn} \right |^2}{E_n^{(0)}-E_m^{(0)}},
\label{rov:aprox:energiePT2}
\end{eqnarray}

\noindent kde $H'_{mn}=\left < \psi_m^{(0)}|H'|\psi_n^{(0)}\right >$. Jistý problém zde představuje skutečnost, že k~výpočtu poruchové energie potřebujeme znát celé energetické spektrum (tj. energii základního i všech excitovaných stavů) energií neporušeného systému. To je v~mnoha případech značně nepraktický požadavek. Pokud víme, jak energie systému přibližně vypadají, můžeme v~některý případech nahradit sumu ve jmenovateli rovnice \ref{rov:aprox:energiePT2} průměrnou excitační energii a dospět k~tzv. uzavírací aproximaci, která dále zjednodušší tvar rovnice pro energii. 

Poslední problém, jemuž jsme se zde nevěnovali jsou degenerované stavy. Pro degenerované stavy poruchová teorie, jak jsme si ji zde formulovali nefunguje, protože člen $E_m^{(0)}-E_n^{(0)}$ bude nulový a vztahy pro rozvojové koeficienty $c_i$ budou divergovat. Pro degenerované vztahy existuje mírně odlišná formulace poruchové teorie, kterou je možné najít v~odborné literatuře. 






