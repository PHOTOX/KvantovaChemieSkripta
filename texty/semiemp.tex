V této kapitole se podíváme na skupinu semiempirických metod. Ačkoli semiempirické metody také vycházejí z řešení elektronové Schr\"{o}dingerovy rovnice (odtud přízvisko \textit{semi}), jejich rovnice obsahují dodatečné parametry, jejichž hodnoty se typicky získávají z přesnějších \textit{ab initio} výpočtů nebo se z experimentálních dat (odtud \textit{empirické}).

Aproximace, které tyto metody využívají, můžeme rozdělit do dvou skupin. V první řadě, většina semiempirických metod nevychází z přesného molekulárního hamiltoniánu XX, ale určitého efektivního hamiltoniánu, ve kterém jsou zanedbány některé členy. Většina semiempirických metod se například soustředí pouze na valenční elektrony (příspěvek ostatních elektronů je brán jako konstanta), což celkový hamiltonián systému značně zjednoduší.

Po určení hamiltoniánu se při odvození typické semiempirické metody postupuje obdobně jako při odvození HF rovnic. Zvolíme si konkrétní tvar vlnové funkce a zvolíme bázi (metoda MO-LCAO) a aplikací variačního principu odvodíme pracovní rovnice. V tomto kroku ale u semiempirických metod dojde k dalšímu zjednodušení. Buď můžeme některé členy z rovnic úplně zanedbat a položit rovno nule, anebo je můžeme považovat za konstantu, jejíž hodnotu určíme tak, aby nám to vycházelo.   

Motivací k vývoji semiempirických metod bylo zmenšení výpočetních nároků \textit{ab initio} metod s pokud možno co nejmenším ovlivněním přesnosti. Nicméně i metody, které byly kvantitativně nepřesné, se ukázaly jako velmi užitečné, neboť dovolily kvalitativní pochopení zákonitostí chemických vazeb a jiných vlastností molekul. Na jednu takovou metodu, která nám dala známé pravidlo \uv{$ 4n+2 $} pro určení aromaticity, se nyní podíváme.

\subsection{H\"{u}ckelova metoda}

H\"{u}ckelova metoda (dále jen HM) je historicky první a asi i nejjednoduší semiempirická metoda. Pro rozumné molekuly je pro použití této metody potřeba doslova jen tužka a papír. Její jednoduchost spočívá v tom, že se soustředí pouze na $\pi$ vazebné elektrony,
tedy elektrony v molekulových orbitalech složených z atomových \textit{p} orbitalů.
Její použití je tudíž omezeno na planární systémy s vícenásobnými vazbami. 

Hamiltonián HM metody má následující tvar:
\begin{equation}
H_{HMO}= \sum h_{i, eff}^\pi
\label{rov:Ham_HM}
\end{equation}
kde $h_{i, eff}^\pi$ je jednoelektronový potencál $\pi$ elektronu, který v sobě obsahuje jak kinetickou energii, tak interakce s ostatními elektrony.
Z toho zřejmé, že elektrony jsou považovány za nezávislé. (to je zásadní rozdíl oproti metodě HF, ve které je nezávislost elektronů vyjádřena \uv{pouze} tvarem vlnové funkce). 
Konkrétní tvar těchto efektivních elektronových hamiltoniánů není specifikován a bere se pouze jako parametr metody. 

Dalším krokem je výběr báze. V případě metody HM to budou $p_z$ orbitaly všech uhlíkových atomů.
\begin{equation}
\psi= \sum_i c_i \phi_i
\label{rov:HM_MO}
\end{equation}
Tím převedeme problém najití vlnové funkce na jednodušší problém najití koeficientů $c_i$. Pokud tuto vlnovou funkci dosadíme do SCHR s Hamiltoniánem \ref{Ham_HM} a aplikujeme variační princip, dostaneme sadu sekulárních rovnic. Z nich poté obdržíme sekulární determinant:
\begin{equation}
|\mathbf{H}-\epsilon \mathbb{S}|=0
\label{rov:HM_det}
\end{equation}

Pro lepší ilustraci budeme nadále uvažovat výpočet pro molekulu butadienu.
Naše báze se tedy bude skládat ze čtyř $p_z$ orbitalů.

Aplikací variačního principu poté dostaneme následující sekulární rovnice:
\begin{equation}
\begin{vmatrix}
H_{11}-\epsilon S_{11} & H_{21}-\epsilon S_{21} & H_{31}-\epsilon S_{31} & H_{41}-\epsilon S_{41}  \\
H_{12}-\epsilon S_{12} & H_{22}-\epsilon S_{22} & H_{32}-\epsilon S_{32} & H_{42}-\epsilon S_{42}  \\
H_{11}-\epsilon S_{13} & H_{23}-\epsilon S_{23} & H_{33}-\epsilon S_{33} & H_{43}-\epsilon S_{43}  \\
H_{11}-\epsilon S_{14} & H_{24}-\epsilon S_{24} & H_{34}-\epsilon S_{34} & H_{44}-\epsilon S_{44}
\end{vmatrix}
= 0
\end{equation}
Nyní bychom měli spočítat integrály $H_{ij}$ a $S_{ij}$, které v principu závisí na geometrii systému.
V rámci H\"{u}ckelovy aproximace na ně ale pohlížíme jako na parametry metody
a naložíme s nimi takto:
\begin{itemize}
\item zanedbáme překryvové integrály mezi AO na různých atomech tj. $S_{ij}=\delta_{ij}$;
\item integrály $H_{ii}=\alpha$ těmto integrálům říkáme coulombovské integrály
\item integrály $H_{ij}$ položíme rovno nule, pokud orbitaly nepatří sousedním atomům.
Nenulové integrály položíme rovno parametru $\beta$ a nazýváme je rezonančními integrály.
\end{itemize}

Takto zjednodušené rovnice poté vyřešíme. Nejprve najdeme vlastní čísla přes sekulární determinant a následně určíme koeficienty $c_i$. V případě butadienu bude sekulární determinant vypadat následovně:
\begin{equation}
\begin{vmatrix}
\alpha-\epsilon & \beta & 0 & 0  \\
\beta &\alpha-\epsilon & \beta & 0  \\
0 &\beta &\alpha-\epsilon & \beta  \\
0 & 0 & \beta &\alpha-\epsilon  
\end{vmatrix}
= 0
\end{equation}
%Tento determinant vede na rovnici čtvrtého stupně, kterou lze ale vhodnou substitucí řešit analyticky.
Ve druhém kroku jsme determinant vydělily $\beta$ a zavedli substituci $x=\frac{\alpha-\beta}{\epsilon}$.
Nyní rozvojem determinantu podle prvního řádku dostaneme:
\begin{equation}
x
\begin{vmatrix}
x & 1 & 0 \\
1 & x & 1 \\
0 & 1 & x \\
\end{vmatrix}
-
\begin{vmatrix}
x & 1 & 0 \\
1 & x & 1 \\
0 & 1 & x \\
\end{vmatrix}
=x^4-3x^2+1=0
\end{equation}
Tuto kvartickou rovnici můžeme naštěstí převést na kvadratickou pomocí substituce $y=x^2$ a dostaneme:
\begin{equation}
y^2-3y+1=0;\quad y_1=2,62;\quad y_2=0.382
\end{equation}
a tedy
\begin{equation}
x_{1,2}=1,62 \nonumber
x_{3,4}=0,62 \nonumber
\end{equation}
Z toho vyplývají násůedující hodnoty energií:

$$ \epsilon_1 = \alpha+1,62\beta $$
$$ \epsilon_2 = \alpha+0,62\beta $$
$$ \epsilon_3 = \alpha-0,62\beta $$
$$ \epsilon_4 = \alpha-1,62\beta $$

Zpětným dosazením do sekulárních rovnic bychom poté dopočítali koeficienty $c_i$ a tím získali molekulové orbitaly. Ty potom obsadíme elektrony v souladu s Pauliho vylučovacím principem a celkovou energii dostaneme jako součet orbitálních energií. 
Jak nyní získáme parametry $\alpha$ a $\beta$? Vyjdeme z experimentálních dat. Parametr $\alpha$ má jednoduchou fyzikální interpretaci; jedná se o ionizační energii atomového orbitalu $p_z$, která je exerimentálně známá. Důležitější je parametr $\beta$, který určuje vzájemnou separaci molekulových orbitalů. Můžeme jej získat v zásadě dvěma způsoby:
\begin{itemize}
\item ze spektroskopických dat. Parametr zvolíme tak, aby nám seděly zvolené elektronové přechody (podrobnosti viz níže). Například fitováním na molekulu naftalenu získáme hodnotu $\beta=3,48$.
\item z termochemických dat. K tomu se využívá konceptu delokalizační energie dvojných vazeb. Tuto energii můžeme experimentálně naměřit například pomocí hydrogenačního tepla. Její teoretický výpočet v rámci HM si ukážeme níže. 
\end{itemize}

\textbf{Spektroskopie metodou HM}

Jak můžeme pomocí HM spočítat excitační energii? Prostě formálně excitujeme jeden elektron z obsazeného do neobsazeného orbitalu a potom spočítáme celkovou energii a od ní odečteme celkovou energii zákaldního stavu. Pokud budeme chtít spočítat nejnižší excitační energii butadienu, přesuneme jeden elektron z orbitalu 3 do orbitalu 2. Výsledná energie potom bude:
$$
\Delta E = \epsilon_3-\epsilon_4 = 1,24 \beta = \frac{hc}{\nu} 
$$
Vlnová délka excitujícího záření potom vyjde 287\,nm, zatímco experimentální hodnota je kolem 200\,nm.
Soulad to není ideální, ale není ani tragický, na to že jsme jej získali tak jednoduše. HM také správně zachycuje experimentální fakt, že excitační energie se snižuje se zvyšující se délkou řetězce obsahující konjugované dvojné vazby.

\textbf{Delokalizační energie}

$$
\left| \right|=0
$$
z čehož nám vyjde
$$
(\alpha-\epsilon)^2=\beta^2
$$
$$
\alpha - \epsilon = \pm \beta
\epsilon_{1,2}= \alpha\pm \beta
$$
Pro butadien poté vyjde delokalizační energie $E_D=0,472\beta=35,4 kJ/mol$
Experimentálně můžeme míru delokalizace určit pomocí hydrogenačních tepel a opět to vztáhnout vůči ethanu.
Hydrogenační teplo ethanu (tj. entalpie reakce ethan + H$_2$ = ethen) je -136,94\,kJ/mol, zatímco hydrogenační teplo butadienu je -240\,kJ/mol. Experimentální delokalizační energie je tedy:
$$
E_D^{exp}=2\cdot139,9-240 = 33,8 kJ/mol,
$$
což je v dobrém souladu s námi vypočtenou hodnotou.

\textbf{Cyklické systémy a aromaticita}
Asi největším úspěchem HM bylo vysvětlení aromaticity cyklických molekul.
Například pro benzen dává HM následující strukturu molekulových orbitalů:
Pokud bychom nyní zkoušeli počítat delokalizační energii pro různý počet elektronů, zjistili bychom, že má maximum pro šest elektronů, tedy pokud plně zaplníme všechny vazebné orbitaly.

Pokud bychom spočítali HM metodou obecně pro cyklické uhlovodíky, došli bychom ke známému H\"{u}ckelovu pravidlu $4n+2$. Z diagramu pro benzen je patrnén, že pro 4n+1 elektronů bychom měli radikál, zatímco pro $4n$ elektronů bychom měli dokonce biradikál.

\begin{priklad}
\textbf{Zadání:} Jak to bude s aromaticitou cyklopropenylového radikálu? A co cyklopentadienylový radikál?

\textbf{Reseni:} V cyklopropenylu máme tři elektrony v $p_z$ orbitalech, stabilnější tedy bude cyklopropenylový kation. Podobně tomu je u cyklopentadienylu máme elektronů pět, stabilnější bude cyklopendienylový anion, který se opravdu často vyskytuje jako ligand v komplexech kovů.
\end{priklad}

\textbf{obrazek}
\bigskip

\subsection{Rozšířená H\"{u}ckelova metoda}

Hlavním vylepšením rožšířené H\"{u}ckelovy metody (RHM) je zavrhnutí $\pi$-elektronové aproximace.
Stále používáme jednoelektronový Hamiltonián a molekulové orbitaly rozvíjíme podobně jako v rovnici \ref{rov:HM_MO}, ale zahrnujeme všechny valenční orbitaly. Atom uhlíku tedy do báze přispěje atomovými orbitaly $2s$, $2p_x$,$2p_y$ a $2p_z$. 
Zde jsou další pravidla pro RHM:

\begin{itemize}
\item Hodnotu Coulombických integrálů opět aproximujeme ionizační energií daného orbitalu.
$$H_{ii}=-I_i$$
\item Hodnoty překryvových integrálů se nezanedbávají, ale počítají se explicitně.
\item Hodnoty rezonančních integrálů se získají z následujícího vztahu:
\begin{equation}
H_{ij}=\frac{1}{2}K(I_i+I_j)S_{ij},
\end{equation}
kde K je empirická konstanta, která má obvykle hodnotu 1,75.
\end{itemize}
Opět řešíme sekulární rovnice XX, ale tentokrát již matice H nebude obsahovat žádné nuly (pokud náhodou není $S_{ij}=0$ ze symetrických důvodů). Tím, že počítáme explicitně překryvové integrály, tak řešení bude záviset na konkrétní geometrii. Tato metoda se tedy dá použít k minimalizaci geometrie, ale výsledky často nejsou příliš kvalitní (například molekula vody vychází jako lineární).  

Podobně jako v původní H\"{u}ckelově metodě tvar pracovních rovnic nezávisí na řešení, a tudíž není třeba iterovat. Rozšířená H\"{u}ckelova metoda se proto i nyní rutinně využívá v kvantově chemických programech k prvotním nástřelu vlnové funkce.

\subsection{Moderní semiempirické metody}

Modernější semiempirické metody již ve svém hamiltoniánu nezanedbávají mezielektronovou repulzi.
Hamiltonián je potom téměř totožný s HF Hamiltoniánem, akorát se soustředíme stále jen na valneční elektrony. Tyto metody poté iterativně řeší Roothanovy rovnice. Největším žroutem výpočetního času jsou dvouelektronové integrály (viz rovnice XX), a proto se právě pro ně zavádí další aproximace.
 
 Pár vět o CNDO (Complete Neglect of Differential Overlap),
 
 INDO (Intermediate Neglect of Differential Overlap),
 
 MINDO (Modified Neglect of Differential Overlap)
 
 AM1(Austin model 1), PM6 atp.

Ačkoli jsou semiempirické zjednodušením původních HF rovnic, často poskytují řešení lepší než HF.
To je dáno tím, že parametry jsou fitovány na experimentální data, a mohou tedy implicitně zahrnout část korelační energie.

Existuje také hojně využívaná semiempirická metoda založená na DFT, takzvaná DFTB (DFT Tight Binding).


 
