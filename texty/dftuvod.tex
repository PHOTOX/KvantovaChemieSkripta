Pomocí metod založených na teorii funkcionálu hustoty(dále jen DFT metod) se v posledních zhruba dvaceti let provádí většina výpočtů elektronové struktury. Popularita DFT metod je dána jejich přijatelnou výpočetní náročností, která o mnoho nepřevyšuje Hartreeho-Fockovu 
metodu. Výpočty jsou ale typicky daleko přesnější, neboť DFT metody zahrnují korelační energii. Efektivita DFT je dána tím, že nepracuje s poměrně složitou vlnovou funkcí (tj. funkcí 3N souřadnic elektronů), ale s tzv. elektronovou hustotou, což je funkce pouze tří prostorových souřadnic. 

Představme si základní pojmy. Začněme pojmem \textbf{funkcionál}. Pojďme si nejprve připomenout, jak funguje funkce. Do funkce vložíme nezávislé proměnou, tedy nějaké číslo, a na oplátku dostaneme číslo jiné, neboli závisle proměnnou. Jde tedy o zobrazení z prostoru (kupř. reálných) čísel opět do prostoru reálných čísel. U funkcionálu je to velmi podobné, akorát na vstupu není číslo, ale funkce. Pokud to tedy řekneme více matematicky, funkcionál je zobrazení z prostoru funkcí na prostor(kupř. reálných) čísel.

Takovým jednoduchým funkcionálem je určitý integrál
$$
F[f(x)] = \int_a^b f(x) \mathrm{d}x, 
$$
Určitý integrál potřebuje dodat vstupní funkci a po jeho vyčíslení dostaneme jedno jediné číslo. Povšimněte si zde zápisu funkcionálu pomocí hranatých závorek.  
Pro složitější příklad funkcionálu nemusíme chodit daleko, stačí se podívat, jak vypadá funkcionál energie v kvantové mechanice.
$$
E[\psi] = \int \psi^*\hat{H}\psi \mathrm{d}\tau .
$$
S některými funkcionály jsme se tak již setkali.

Mezi funkcemi a funkcionály existují i další podobnosti. Pojmy jako minimum a maximum funkcionálu mají prakticky stejný význam. Existuje i funkcionální analogie k dobře známé derivaci funkce --- mluvíme o \textbf{variaci funkcionálu}.
Když děláme derivaci funkce, tak se vlastně díváme, co se děje se závisle proměnnou, když trochu změníme nezávisle proměnnou.
U variace je to podobné. Zajímá nás, jak se změní hodnota funkcionálu, když mírně změníme naši funkci. Přesná definice je složitější a neuvádíme ji zde. Bude ale pro nás důležité, že pro variace platí podobné vztahy, jako pro derivace. Dá se například ukázat, že v minimu funkcionálu je jeho variace nulová. Již dobře známý variační princip (konečně víme, proč se mu tak říká!) pak můžeme napsat jednoduše pomocí variace tohoto funkcionálu energie jako
$$
\delta E[\Psi] = 0 .
$$

Nyní se můžeme vrátit k definici ústřední veličiny této kapitoly --- elektronové hustoty $\rho(\mathbf{r})$.
Elektronová hustota má na rozdíl od vlnové funkce přímý fyzikální význam. Jedná se o pravděpodobnost, že v nějakém bodě prostoru najdeme \textbf{nějaký elektron}. Je důležité si uvědomit rozdíl mezi elektronovou hustotou a čtvercem vlnové funkce, který taktéž udává hustotu pravděpodobnosti. Čtverec vlnové funkce nám udává pravděpodobnost, že první elektron má spin $m_{s1}$ a je v bodě $\mathbf{r}_1$, druhý elektron má spin $m_{s2}$ a je v bodě ($\mathbf{r}_2$) atd. Jedná se tedy o mnohem složitější veličinu, která závisí na celkem $4N$ proměnných, zatímco elektronová hustota závisí jen na třech proměnných.

Elektronová hustota souvisí s vlnovou funkcí systému dle vztahu
\begin{equation}
\rho=N \int |\psi(\textbf{r}_1,\textbf{r}_2,...,\textbf{r}_n)|^2 \mathrm{d}\textbf{r}_2\dots\mathrm{d}\textbf{r}_n .
\end{equation}
\textbf{Chce to okomentovat, proc tomu tak je} Z této definice hned plyne několik důležitých vlastností.

\begin{itemize}
\item Elektronová hustota je nezáporná veličina, platí tedy
\begin{equation}
\rho(\mathbf{r})  > 0 .
\end{equation}
\item Pokud zintegrujeme elektronovou hustotu přes celý prostor, dostaneme počet elektronů v systému jako
\begin{equation}
\int \rho\mathrm{d}r = N
\end{equation}

\item V poloze jader má elektronová hustota maxima.
\item Pro tato maxima platí (XXX teorém), že
\begin{equation}
\lim_{r_i \to 0} \left[ \frac{\delta}{\delta r}+2Z_A\right]\hat{\rho(r)}=0, 
\end{equation}
kde $\hat{\rho}$ je angulárně zprůměrovaná hodnota elektronové hustoty a $Z_A$ je náboj příslušného atomového jádra.
\end{itemize}

Z těchto vlastností vidíme, že pokud známe elektronovou hustotu, tak zároveň také můžeme zjistit počet elektronů, polohu jader i jejich náboj. To jsou ale přesně ty informace, které jsou potřeba ke specifikaci molekulárního hamiltoniánu
\begin{equation}
\hat{H}=\sum_{i=1}^N -\frac{1}{2}\Delta_i+\sum_{i=1}^N\sum_{j=i+1}^N\frac{1}{r_{ij}}-\sum_{i=1}^N\sum_{A=1}^K \frac{Z_A}{r_{iA}} ,
\label{rov:ham_dft}
\end{equation}
kde jsme vynechali pro jednoduchost člen popisující odpuzování jader, jenž je stejně v rámci Bornovy--Oppenheimerovy aproximace roven konstantě. Pokud ale elektronová hustota jednoznačně určuje hamiltonián, tak poté z řešení SCHR taky energii a vlnovou funkci, a tudíž všechny potřebné veličiny. Podobným způsobem se zřejmě ubíraly úvahy Hohenberga a Kohna, kteří postavili teorii DFT na pevné fyzikální základy.

