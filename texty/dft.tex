Jak již bylo řečeno v úvodu k minulé kapitole, teorie funkcionálu hustoty (dále jen DFT) je zásadně odlišná od jiných metod kvantové chemie, neboť nepracuje s elektronovou vlnovou funkcí, ale s elektronovou hustotou. DFT metody se staly v posledních zhruba 15 letech obrovsky populární, jelikož přesností leckdy odpovídají korelovaným metodám, ale výpočetní náročnost odpovídá zhruba metodě Hartreho--Focka. Dříve než se pustíme do této zajímavé teorie, musíme si uvést některé nezbytné matematické pojmy. 

Nejprve si musíme říct, co je to vlastně ten funkcionál.

Nejjednodušším příkladem funkcionálu je
$$
F[f(x)] = \int f(x) \mathrm{d}x, 
$$
nejedná se o nic jiného než starý dobrý určitý integrál! 
Pro složitější příklad funkcionálu nemusíme chodit daleko, stačí se podívat, jak vypadá funkcionál energie v kvantové mechanice.
$$
E[\psi] = \int \psi^*\hat{H}\psi \mathrm{d}\tau .
$$
Vidíme, že jsme se s funkcionály již setkali, akorát nám autoři těchto skript zamlčeli.

Podobně jako funkce můžeme derivovat, podobná operace existuje i pro funkcionály, říká se jí variace funkcionálu.
Když děláme derivaci funkce, tak se vlastně díváme, co se děje se závisle proměnnou, když trochu změníme nezávisle proměnnou.
U variace je to podobné. Zajímá nás, jak se změní hodnota funkcionálu, když mírně změníme naši funkci. Přesná definice je složitější a neuvádime ji zde. Bude ale pro nás důležité, že pro variace platí podobné vztahy, jako pro derivace. Dá se například ukázat, že v minimu funkcionálu je jeho variace nulová. Již dobře známý variační princip (konečnš víme, proč se mu tak říká!) pak můžeme napsat jednoduše pomocí variace tohoto funkcionálu energie jako
$$
\delta E[\Psi] = 0
$$

Nyní již můžeme přejít k definici ústřední veličiny této kapitoly --- elektronové hustoty $\rho(\mathbf{r}$.
Elektronová hustota má narozdíl od vlnové funkce přímý fyzikální význam. Jedná se o pravděpodobnost, že v nějakém bodě prostoru najdeme \textbf{elektron}. Je důležité si uvědomit rozdíl mezí elektronovou hustotou a čtvercem vlnové funkce, který taktéž udává hustotu pravděpodobnosti. 

Elektronová hustota souvisí s vlnovou funkcí systému následovně
\begin{equation}
\rho=N \int |\psi(\textbf{r}_1,\textbf{r}_2,...,\textbf{r}_n)|^2 \mathrm{d}\textbf{r}_2\dots\mathrm{d}\textbf{r}_n .
\end{equation}
Z této definice hned plyne několik důležitých vlastností.

\begin{itemize}
\item Elektronová hustota je nezáporná veličina.
\begin{equation}
\rho(\mathbf{r})  > 0
\end{equation}
\item Pokud zintegrujeme elektronovou hustotu přes celý prostor, dostaneme počet elektronů v systému.
\begin{equation}
\int \rho\mathrm{d}r = N
\end{equation}

Platí také následující dva vztahy.

\item V poloze jader má elektronová hustota maxima.
\item Pro toto maximum platí, že
\begin{equation}
\lim_{r_i \to 0} \left[ \frac{\delta}{\delta r}+2Z_A\right]\rho(r)=0
\end{equation}

\end{itemize}


\subsection{Hohenbergovy--Kohnovy teorémy}



2. HK teorém 
Předpokládejme, že pro daný externí potenciál $\nu_{ext,0}$ je správná elektronová hustota $\rho_0$. Pak pro jakoukoli funkci $\rho$ bude platit: $$ E[\rho_0] < E[\rho] $$

Důkaz:
Z prvního HK teorému plyne, že každá funkce $\rho$ jednoznačně definuje externí potenciál (který je odlišný od $v_{ext,0}$), a tudíž i nějakou vlnovou funkci $\psi$. Pokud ale pro tuto vlnovou funkci vyčíslíme energii, pak nám z již známého variačního principu plyne:
\begin{equation}
E^\prime=\int \psi^{\prime *} \hat{H} \psi^{\prime} \mathrm{d}\tau = E[\rho] > E_0 = E_0 [\rho_0] ,
\end{equation}
což jsme chtěli dokázat.

\subsection{Kohnovy--Shamovy rovnice}

Pojdme si napsat ještě jednou elektronový Hamiltonián molekuly s N elektrony. Vynecháme člen popisujicí repulzi jader, jelikož ten je stejně v rámci BO aproximace konstantní. 
\begin{equation}
\hat{H}=\sum_{i=1}^N -\frac{1}{2}+\sum_{i=1}^N\sum_{j=i+1}^N\frac{1}{r_{ij}}-\sum_{i=1}^N\sum_{A=1}^K \frac{Z_A}{r_{iA}}
\end{equation}

Formálně můžeme funkcionál energie napsat následovně:
\begin{equation}
E=\int \psi^*\hat{H}\psi \mathrm{d}\tau = \int \psi^*\hat{T}\psi\mathrm{d}\tau \int \psi^*\hat{V_{el}}\psi\mathrm{d}\tau \int \psi^*\nu_{ext}\psi\mathrm{d}\tau=T[\rho]+V_{ee}[\rho]+V_{ne}[\rho]
\end{equation}
Funkcionál energie jsme tedy rozdělili na funkcionál kinetické energie elektronů, funkcionál repulze elektronů a funkcionál interakce elektronů s jádry. Jelikož externí potenciál $\nu_{ext}$ nezávisí na polohách elektronů, můžeme si posledně jmenovaný funkcionál jednoduše přepsat do tvaru, ve kterém již vystupuje elektronová hustota:
\begin{equation}
V_{ne}=\int \psi^*\nu_{ext}\psi\mathrm{d}\tau = \int \rho(\textbf{r})\nu_{ext}(\textbf{r}) \mathrm{d}\textbf{r} 
\end{equation} 




\subsection{Aproximace lokální hustoty}

LDA (Local density Approximation)
\begin{equation}
E_{xc}^{LDA}=\int \rho(\textbf{r})V_{xc}(\rho(\textbf{r}))\mathrm{d}\textbf{r} 
\end{equation}


\begin{equation}
V_{xc}(\rho)=V_x(\rho)+V_c(\rho)
\end{equation}
Pro výměnný člen platí pro homogenní plyn následující vztah:
\begin{equation}
V_x(\rho)=-\frac{3}{4}\left(\frac{3}{\pi}\right)^{\frac{1}{3}}\rho^{\frac{1}{3}}
\end{equation}
Pro korelační energii UEG nelze získat analytický výraz. Příslušné výpočty lze ale provést numericky a výsledek poté nafitovat. Výsledný korelační funkcionál je znám jako VWN (dle pánů Vosko-Wilk-Nusair).

LSDA....$E_{xc}=E_{xc}[\rho_\alpha\rho_\beta]$ 


\subsection{GGA a hybridní funkcionály}
GGA 
\begin{equation}
E_{xc}^{GGA}=\int \rho(\textbf{r})f(\rho,\Delta\rho\mathrm{d}\textbf{r} 
\end{equation}

Becke: B88, LYP (Lee-Yang-Parr),
kombinace: BP86, BLYP, PBE

meta-GGA

\textbf{Hybridní funkcionály}
\begin{equation}
E_X^{exact}=-\frac{1}{2}\sum_{i=1}^N\sum_{j=1}^N K_{ij}
\end{equation}

B3LYP:
\begin{equation}
E_{xc}^{B3LYP}=(1-a_0-a_x)E_x^{LDA}+a_0E_x^{exact}+a_xE_x^{B88}+(1-a_c)E_c^{VWN}+a_c E_c^{LYP}
\end{equation}
$a_0=0,2$; $a_x=0,72$; $a_c=0,81$

PBE0 (25\,\%), BMK,BHandHLYP(50\,\%)

\textbf{Long-range corrected funkcionály}
Mělo by platit, že $lim_{V_x\to \infty}=\frac{1}{r}$

\textbf{Disperzní korekce}

Grimmeho empirická korekce
\begin{equation}
E_{vdw}=-s_6\sum c^{ij}r_{ij}^{-6}
\end{equation}
Koeficient $s_6$ závisí na použitém funkcionálu, zatímco koeficienty $c_{ij}$ závisí na typu interagujících atomů.

\textbf{Dvojitě hybridní funkcionály}
S.Grimme 
\begin{equation}
E_{xc}^{hybrid}=a_1E_x^{GGA}+(1-a_1)E_x^{EXACT}+a_2E_c^{GGA}
\end{equation}
Pro tento tvar se vyřeší KS rovnice a získají KS orbitaly. Z těchto orbitalů se poté vypočítá MP2 korekce
ze vzorečku XX a přídá se k $E_{xc}^{hybrid}$

\begin{equation}
E_{xc}^{DH}=E_{xc}^{hybrid}+(1-a_2)E_c^{KS-MP2}
\end{equation}

Příkladem může být B2LYP s parametry $a_1=0,47$ a $a_2$=0,73

% Jakobův žebřík, obrázek (Genesis 28:10-12)